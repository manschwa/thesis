% class options:
% - select either [german] or [english]
% - select the type of thesis from:
%   [bachelor, master, generic]
%   (in case of generic, use \type{} to specify it)
% - use option "alpha" for abbreviated citation (instead of numbers)
% - option "draft" is available, too
% - use options "utf8" or "latin1" to select inputencoding
\documentclass[german, master, latin1]{base/thesis}

\usepackage{units}    % useful for settings units:              \unit[23]{m}
\usepackage{nicefrac} % for setting fractions esp. within text: \nicefrac{km}{h}

\usepackage{algorithm, algorithmic}  % for pseudo code (cf. documentation)
\renewcommand{\algorithmiccomment}[1]{\qquad{\small // \textit{#1}}}

%%%%%%%%%%%%%%%%%%%%%%%%%%%%%%%%%%%%%%%%%%%%%%%%%%%%%%%%%%%%%%%%%%%%%%%%%%%%%%%

\begin{document}

\title{Anbindung von Messaging-Systemen an Lernmanagementsysteme (am Beispiel von Stud.IP und Matrix)}
\author{Manuel Schwarz}
\email{manschwa@uos.de}
\firstSupervisor{Dr. Tobias Thelen}
\secondSupervisor{Prof. Dr. Elke Pulverm�ller}
%\shorttitle{...}                       % by default = title
%\dept{...}                             % by default KBS UOS
%\submitdate{November 2004}             % by default current month & year
%\signcity{}                            % by default Osnabr�ck
%signline{Osnabr�ck, 11. Dezember 2004} % by default "signcity, submitdate"

\generatetitle

\cleardoublepage

\begin{prefacesection}{Zusammenfassung}

Die vorliegende Arbeit \dots

\end{prefacesection}

\cleardoublepage
\tableofcontents


\startTextChapters %%%%%%%%%%%%%%%%%%%%%%%%%%%%%%

\chapter{Motivation}

Lehren und Lernen im digitalen Zeitalter.\\
Moderne Kommunikationswege und - mittel studiumsunterst�tzend einsetzen.\\
Aktuelle Werkzeuge nutzen, um Studierenden eine m�glichst niedrige
Einstiegsschwelle bei Fragen oder Unklarheiten zu bieten.\\
Fortw"ahrender Prozess der Weiterentwicklung und Anpassung der Vermittlung von Informationen.\\
Mailinglisten, Foren, Instant-Messenger \dots \\
Wie studiert man heute? Wie ist die durchschnittliche Nutzung der
Studierenden von WhatsApp und Co.? (Bezug auf die Studien)\\




\chapter{Anforderungen}
Kernpunkt der Arbeit, auf dem der Hintergrund und die Implementation aufbauen.
Empirischer Teil: Anforderungsanalyse mit Hilfe eines (Mini-) Fragebogens.

Unterkapitel f�r jede herausgearbeitete Anforderung (funktional + nicht funktional).

\section{Veranstaltungen auf R"aume abbilden}
Kurse, Studengruppen, \dots sollten (automatisch) auf R"aume in Matrix abgebildet werden.

\section{kein zus"atzliches Login}
Nur ein Login f�r beide Systeme/Dienste. LDAP-Anbindung. Stud.IP Nutzer sollten ohne zus"atzliches Login
in einen Matrix Raum gelangen k�nnen.

\section{�bernahme der Teilnehmendenlisten}

\section{�bersichtliches UI}
Klare �bersicht (Nachrichten und Bedienung) undabh"angig von OS/Browser/Screensize (Portabilit"at).
Gute Usability, wenig Klicks und ein modernes Erscheinungsbild.

\section{Benachrichtigungsfunktion}
Eher ungeeignet: unibezogene/veranstaltungs�bergreifende Informationen, wie z.B. R�ckmeldefristen, Unischlie�ung, \dots
(dies w"are ein reiner Infochannel, in dem alle Studierenden sein m�ssten,
keine Antwortm�glichkeit -> one-way Kommunikation -> eMail/Stud.IP Benachrichtigung wie bisher ist da vielleicht besser)
Besser: veranstaltungsbezogene Informationen, z.B. "eine Datei wurde hochgeladen", "Termin f"allt aus"

\section{Nachrichten senden und empfangen (Basisfunktionen)}

\section{Direktnachrichten}
1-zu-1 Chats

\section{Transparenz}
Verkn�pfung zwischen Stud.IP und Matrix/Element deutlich machen.

\section{�bernahme von Rollen und Rechten}
Feingranulare Berechtigungen f�r Studierende ist nur eingeschr"ankt umsetzbar, da es in Matrix nur 3 Rechtestufen gibt.
Denkbar w"are grob: Dozenten -> Moderator/Admin und Studierende -> Default.
In Matrix lassen sich zus"atzlich benutzerdefinierte Rechtestufen erstellen, eventuell lassen sich damit alle in Stud.IP
verf�gbaren Rechtestufen abbilden.

\section{mehrere R"aume pro Kurs}
Es sollten aus einer Stud.IP Veranstaltung mehrere Matrix-R"aume erstellt werden k�nnen, die dann als Community
zusammengefasst werden k�nnten.

\section{Synchronizit"at}
Zum einen sollten Matrix und Blubber zeitlich synchron sein. Zum anderen w"are es w�nschenswert, wenn
die in einem System als gelesen markierte Nachrichten auch im anderen System als gelesen markiert w�rden.

\section{Matrix/Element in Blubber verlinken}
Chat/Matrix/Element-Icon in Blubber anzeigen, mit dem man zu dem Uni-Matrix/Element-Raum gelangt.

\section{Matrix als Hauptmessengingdienst}
Weg von Blubber.

\section{Kompatibilit"at von Nachrichten in Blubber und Matrix}
\subsection{Dateien versenden}
Verlinken von Inhalten in Stud.IP?
Eventuell zu aufw"andig.
\subsection{Emoticons korrekt anzeigen}
\subsection{Nachrichten sollten editier- und l�schbar sein}

\section{Opensource}
\section{Stud.IP Plugin}
\section{keine zus"atzliche API}
\section{Sicherheit}



\chapter{Hintergrund}

\section{Lernmanagementsysteme}
Was ist das?
\subsection{Stud.IP}
Konkretes Beispiel eines LMS
\subsection{Stud.IP Schnittstellen}
JSON-API (Blubber)
https://hilfe.studip.de/develop/Entwickler/JSONAPI

\section{Messenger}
Was ist das?
\subsection{Matrix}
Konkretes Beispiel eines Messenger-Backends.
\subsection{Matrix Schnittstellen}
https://matrix.org/docs/guides/client-server-api




\chapter{Implementation}

\section{Mappen der APIs}



\chapter{Evaluation}
Da vermutlich keine Zeit f�r eine empirische Evaluation mit einem IRL Test + Umfrage bleibt,
werden hier technische Tests hinreichen m�ssen.

\section{Technische Tests}
\subsection{Unit-Tests}
Testbeschreibung
\subsection{Code Coverage}
\section{Chancen}
\section{Probleme}


\chapter{Ausblick}
Potentielle Nutzung in der Zukunft und Weiterentwicklung?


\newpage
Beispieltext. Siehe auch \cite{Schulmeister:2005, Besl:1992, Thrun:2000}.
URLs gehen auch: \cite{your-study}.

% DON'T set \bibliographystyle here -- use the documentclass option instead
\bibliography{papers}

\closing %%%%%%%%%%%%%%%%%%%%%%%%%%%%%%

\end{document}
