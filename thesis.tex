% class options:
% - select either [german] or [english]
% - select the type of thesis from:
%   [bachelor, master, generic]
%   (in case of generic, use \type{} to specify it)
% - use option "alpha" for abbreviated citation (instead of numbers)
% - option "draft" is available, too
% - use options "utf8" or "latin1" to select inputencoding
\documentclass[german, master, latin1]{base/thesis}

\usepackage{units}    % useful for settings units:              \unit[23]{m}
\usepackage{nicefrac} % for setting fractions esp. within text: \nicefrac{km}{h}

\usepackage{algorithm, algorithmic}  % for pseudo code (cf. documentation)
\renewcommand{\algorithmiccomment}[1]{\qquad{\small // \textit{#1}}}

%%%%%%%%%%%%%%%%%%%%%%%%%%%%%%%%%%%%%%%%%%%%%%%%%%%%%%%%%%%%%%%%%%%%%%%%%%%%%%%

\begin{document}

\title{Anbindung von Messaging-Systemen an Lernmanagementsysteme (am Beispiel von Stud.IP und Matrix)}
\author{Manuel Schwarz}
\email{manschwa@uos.de}
\firstSupervisor{Dr. Tobias Thelen}
\secondSupervisor{Prof. Dr. Elke Pulverm�ller}
%\shorttitle{...}                       % by default = title
%\dept{...}                             % by default KBS UOS
%\submitdate{November 2004}             % by default current month & year
%\signcity{}                            % by default Osnabr�ck
%signline{Osnabr�ck, 11. Dezember 2004} % by default "signcity, submitdate"

\generatetitle

\cleardoublepage

\begin{prefacesection}{Zusammenfassung}
Die vorliegende Arbeit \dots
\end{prefacesection}

\cleardoublepage
\tableofcontents


\startTextChapters %%%%%%%%%%%%%%%%%%%%%%%%%%%%%%

\chapter{Motivation}



\chapter{Hintergrund}

\section{E-Learning}
\section{Lernmanagementsysteme}
\section{Moderne Kommunikationswege / Kommunikation in der Digitalen Lehre}
Wie kommunizieren die Studierenden haupts"achlich im Zeitalter der Digitalen Lehre?
\section{Messenger}



\chapter{Anforderungen}

\section{Opensource}
\section{Messengerauswahl - Softwarealternativen}
\begin{itemize}
    \item{Matrix}
    \item{Rocketchat}
    \item{Mattermost}
\end{itemize}



\chapter{Implementation}

\section{Stud.IP}
\subsection{Blubber}
\subsection{JSON-API}

\section{Matrix}
\subsection{Synapse}
\subsection{Client - Element}

\section{Mappen der APIs}



\chapter{Evaluation}

\section{Chancen}
\section{Probleme}


\chapter{Ausblick}
Potentielle Nutzung in der Zukunft und Weiterentwicklung?


\newpage
Beispieltext. Siehe auch \cite{Fischer:1995, Besl:1992, Thrun:2000}.
URLs gehen auch: \cite{robocup}.

% DON'T set \bibliographystyle here -- use the documentclass option instead
\bibliography{papers}

\closing %%%%%%%%%%%%%%%%%%%%%%%%%%%%%%

\end{document}
